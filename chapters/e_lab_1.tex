\chapter{Electronic Lab}

\section{Active Band-pass Filter}

\subsection{Task Description}

    A second-order active bandpass filter was designed using the LM4562 dual operational amplifier by cascading two staggered tuned first-order bandpass stages in multi-feedback topology. 
    The transfer function and component values were calculated manually according to the given specifications (Butterworth and 1 dB Chebyshev, 4–5 kHz, 20 dB gain). E24-series components with \SI{1}{\percent} tolerance were chosen.
    
    The circuit was simulated in PSpice (transient, AC, noise and Monte Carlo analysis), implemented on a prototype board with ±5 V supply, and experimentally characterized. 
    Frequency response, passband gain, 3 dB corner frequencies and step response were measured and compared with simulation results in a combined Bode plot.
    
    Finally, the output noise behavior was analyzed and discussed in the report.

\subsection{Schematic}
    \begin{figure}[H]
        \centering
        \includegraphics[width=0.9\linewidth]{img//e_lab_1/schematic-2_2.png}
        \caption{ Schematic of the second order band pass filter.}
        \label{fig:Schematic-2_2}
    \end{figure}

\subsection{Formulas and Calculations}
    The calculations were done by hand. A scan of the hand written notes can be found in section \ref{sec:calc_active_band_pass} of the appendix
 
\subsection{Table(s) with Measurement Results}
    \begin{table}[H]
        \centering
        \caption{Component values for Butterworth band-pass filter}
        \begin{tabular}{c!{\vrule width 2pt}c!{\vrule width 2pt}c!{\vrule width 2pt}c|}
            \cline{2-4}
             & \textbf{Calculated} & \textbf{Chosen} & \textbf{Measured} 
            \\ \noalign{\hrule height 2pt}
            \multicolumn{1}{|c!{\vrule width 2pt}}{$R_1$ in k\textohm} & 5.448 & 5.6 & 5.5831 \\ \hline
            \multicolumn{1}{|c!{\vrule width 2pt}}{$R_2$ in k\textohm} & 48.880 & 47 & 46.943 \\ \hline
            \multicolumn{1}{|c!{\vrule width 2pt}}{$R_3$ in \textohm} & 303.59 & 300 & 298.59 \\ \hline
            \multicolumn{1}{|c!{\vrule width 2pt}}{$R_4$ in k\textohm} & 4.650 & 4.7 & 4.6045 \\ \hline
            \multicolumn{1}{|c!{\vrule width 2pt}}{$R_5$ in k\textohm} & 41.718 & 43 & 42.798 \\ \hline
            \multicolumn{1}{|c!{\vrule width 2pt}}{$R_6$ in \textohm} & 259.11 & 270 & 269.01 \\ \hline
            \multicolumn{1}{|c!{\vrule width 2pt}}{$C_1$ in nF} & - & 10 & 10.03 \\ \hline
            \multicolumn{1}{|c!{\vrule width 2pt}}{$C_2$ in nF} & - & 10 & 10.09 \\ \hline
            \multicolumn{1}{|c!{\vrule width 2pt}}{$C_3$ in nF} & - & 10 & 10.03 \\ \hline
            \multicolumn{1}{|c!{\vrule width 2pt}}{$C_4$ in nF} & - & 10 & 10.10 \\ \hline
        \end{tabular}
        \label{tab:butterworth_components}
    \end{table}
    
    \begin{table}[H]
        \centering
        \caption{Component values for Chebyshev 1 dB band-pass filter}
        \begin{tabular}{c!{\vrule width 2pt}c!{\vrule width 2pt}c!{\vrule width 2pt}c|}
            \cline{2-4}
             & \textbf{Calculated} & \textbf{Chosen} & \textbf{Measured} 
            \\ \noalign{\hrule height 2pt}
            \multicolumn{1}{|c!{\vrule width 2pt}}{$R_1$ in k\textohm} & 7.532 & 7.5 & - \\ \hline
            \multicolumn{1}{|c!{\vrule width 2pt}}{$R_2$ in k\textohm} & 124.641 & 120 & - \\ \hline
            \multicolumn{1}{|c!{\vrule width 2pt}}{$R_3$ in \textohm} & 117.89 & 120 & - \\ \hline
            \multicolumn{1}{|c!{\vrule width 2pt}}{$R_4$ in k\textohm} & 6.492 & 6.2 & - \\ \hline
            \multicolumn{1}{|c!{\vrule width 2pt}}{$R_5$ in k\textohm} & 107.432 & 110 & - \\ \hline
            \multicolumn{1}{|c!{\vrule width 2pt}}{$R_6$ in \textohm} & 101.613 & 100 & - \\ \hline
            \multicolumn{1}{|c!{\vrule width 2pt}}{$C_1$ in nF} & - & 10 & 10.03 \\ \hline
            \multicolumn{1}{|c!{\vrule width 2pt}}{$C_2$ in nF} & - & 10 & 10.09 \\ \hline
            \multicolumn{1}{|c!{\vrule width 2pt}}{$C_3$ in nF} & - & 10 & 10.03 \\ \hline
            \multicolumn{1}{|c!{\vrule width 2pt}}{$C_4$ in nF} & - & 10 & 10.10 \\ \hline
        \end{tabular}
        \label{tab:chebyshev_components}
    \end{table}
    The measured resistor values show deviations below 1\% with respect to the nominal chosen E24 values, which confirms the specified tolerance class. 
    The capacitor measurements also confirm the 1\% NP0 specification and therefore only introduce minor shifts in the center frequency.
    
    \begin{table}[H]
        \centering
        \caption{Center and corner frequencies with their corresponding gain values.}
        \begin{tabular}{c!{\vrule width 2pt}c!{\vrule width 2pt}c!{\vrule width 2pt}c|}
            \cline{2-4}
             & \textbf{-3dB Corner Frequency} & \textbf{Center Frequency} & \textbf{+3dB Corner Frequency} 
            \\ \noalign{\hrule height 2pt}
            \multicolumn{1}{|c!{\vrule width 2pt}}{Frequency in Hz} & 4.32 & 4.57 & 4.94 \\ \hline
            \multicolumn{1}{|c!{\vrule width 2pt}}{Gain in dB} & 19.3 & 22.3 & 19.3 \\ \hline
        \end{tabular}
        \label{tab:corner_and_center_frequency_gain}
    \end{table}

    \noindent The measurements for the bode plot can be found at the appendix in table \ref{tab:bode_measurements_active_band_pass}.

\subsection{Curves \& Diagrams}
    \begin{figure}[H]
        \centering
        \includesvg[inkscapelatex=false, width=0.7\linewidth]{img/e_lab_1/pic9_sine_wave}
        \caption{\centering Transient simulation showing the sinusoidal input voltage and the output voltage of each of the two amplifier stages.}
        \label{fig:pic_9_transient_sine}
    \end{figure}
    
    \begin{figure}[H]
        \centering
        \includesvg[inkscapelatex=false, width=0.7\linewidth]{img/e_lab_1/pic10_step_response}
        \caption{\centering Transient simulation showing the step response of both amplifier stages after applying a signal transitioning from \SI{0}{V} to \SI{1}{V} to the input.}
        \label{fig:pic_10_step_response}
    \end{figure}
    
    \begin{figure}[H]
        \centering
        \includesvg[inkscapelatex=false, width=0.7\linewidth]{img/e_lab_1/pic11_noise_density}
        \caption{\centering Simulation results of the noise spectral voltage density.}
        \label{fig:pic_11_noise_density}
    \end{figure}

    \begin{figure}[H]
        \centering
        \includesvg[inkscapelatex=false, width=0.7\linewidth]{img/e_lab_1/pic12_total_noise}
        \caption{\centering Simulation results of the total output noise voltage.}
        \label{fig:pic_12_total_noise}
    \end{figure}
    
    \begin{figure}
        \centering
        \includegraphics[width=1\linewidth]{img/e_lab_1/pic13_monte_carlo_3db_bandwidth.png}
        \caption{\centering Monte Carlo simulation of the \SI{3}{db} bandwidth. \SI{1}{\%} tolerances were used for both resistors and capacitors.}
        \label{fig:pic_13_monte_carlo_3db}
    \end{figure}
    
    \begin{figure}
        \centering
        \includegraphics[width=1\linewidth]{img/e_lab_1/pic14_monte_carlo_center_frequency_20db.png}
        \caption{\centering Monte Carlo simulation of the center frequency. \SI{1}{\%} tolerances were used for both resistors and capacitors.}
        \label{fig:pic_14_monte_carlo_center_frequency}
    \end{figure}
    
    \begin{figure}
        \centering
        \includegraphics[width=1\linewidth]{img/e_lab_1/pic13_monte_carlo_3db_bandwidth.png}
        \caption{\centering Monte Carlo simulation of the gain variation. \SI{1}{\%} tolerances were used for both resistors and capacitors.}
        \label{fig:pic_15_monte_carlo_gain}
    \end{figure}
    
    \begin{figure}
        \centering
        \includegraphics[width=1\linewidth, trim=0 2cm 0 0, clip]{img//e_lab_1/Picture 17.png}
        \caption{\centering Measurement of the filters step response. A \SI{0,5}{V} square wave was used as the step function.}
        \label{fig:pic_17_step_response}
    \end{figure}
    
    \begin{figure}
        \centering
        \includegraphics[width=0.8\linewidth, trim=0 2cm 0 0, clip]{img//e_lab_1/Picture 16-f0-max.png}
        \caption{Measurement of the passband gain.}
        \label{fig:pic_16_passband_gain}
    \end{figure}
    
    \begin{figure}[H]
        \centering
        \includesvg[inkscapelatex=false, width=1\linewidth]{img/e_lab_1/pic11_bode}
        \caption{\centering Bode plot displaying both simulated and measured values.}
        \label{fig:bode_1}
    \end{figure}

    \begin{figure}[H]
        \centering
        \includesvg[inkscapelatex=false, width=0.6\linewidth]{img/e_lab_1/calculated_vs_actual_res_values}
        \caption{\centering Simulation results of the circuits magnitude response when using calculated vs. actual resistor values.}
        \label{fig:pic_calculated_vs_actual_res_values}
    \end{figure}

\subsection{Discussion of Measurement Results}
    While simulated and measured phase and magnitude responses show very good agreement (figure \ref{fig:bode_1}), the measured passband gain of \SI{22.3}{dB} 
    is slightly higher than the expected \SI{20}{dB}, both in the simulated and measured results. 
    This discrepancy is caused by the fact, that the calculated resistor values are not available as standard E24 values and therefore had to be approximated.
    The impact of this approximation was analyzed using a PSpice simulation. The simulation results can be seen in figure \ref{fig:pic_calculated_vs_actual_res_values}
    and show that when using the calculated resistor values, the expected passband gain of \SI{20}{dB} is achieved, while the gain is increased when using the actual measured resistor values.
    
    The simulation and measurements also show that the lower \SI{3}{dB} corner frequency is about \SI{300}{Hz} higher than the expected \SI{4}{kHz}, while the upper \SI{3}{dB} 
    corner frequency deviates only slightly from the expected value of \SI{5}{kHz} (table \ref{tab:corner_and_center_frequency_gain}). The reason for this shift are the capacitors component values, which were predefined.
    This deviation from the expected values is also visible in the monte carlo simulation of the \SI{3}{dB} bandwidth (figure \ref{fig:pic_13_monte_carlo_3db}), which shows a mean value of \SI{686}{Hz} which is about \SI{300}{Hz} lower than the expected value of \SI{1000}{Hz}.
    The monte carlo simulation of the center frequency shows a shift of the center frequency as well. The expected center frequency is \SI{4.47}{kHz} (geometric mean) while the mean value of the monte carlo simulation is \SI{4.76}{kHz} (figure \ref{fig:pic_14_monte_carlo_center_frequency}).

    To measure the output noise voltage of the circuit, the input has to be connected to ground. The output signal then has to be amplified using a suitable amplifier stage and can afterwards be measured with an oscilloscope using its RMS measurement function.