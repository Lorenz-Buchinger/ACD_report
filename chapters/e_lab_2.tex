\section{ADC-Driver And Anti-aliasing Filter}

\subsection{Task Description}
    A single-ended to differential driver for the AD7626 ADC was designed using the LM4562 dual operational amplifier. 
    The circuit includes a second-order active low-pass filter (100 Hz – 40 kHz) with a total differential gain of 20 and a DC offset of 1 V at both outputs.
    
    The circuit was simulated in PSpice (Bode plot, time-domain response and output noise), implemented on a prototype board, and experimentally characterized. 
    Magnitude and phase response, step response, and differential output signals were measured and compared with the simulation results.
    
    Finally, the SNR and ENOB were calculated for a full-scale ADC input signal.

\begin{table}[H]
    \centering
    \begin{tabular}{|c|c|}
        \hline
        $C_1$ & \SI{150}{pF} \\ \hline
        $C_2$ & \SI{4,7}{nF} \\ \hline
        $f_{cl}$ & \SI{100}{Hz} \\ \hline
        $f_{cu}$ & \SI{40}{kHz} \\ \hline
        $A_0$ & 20 \\ \hline
    \end{tabular}
    \caption{Given values.}
    \label{tab:placeholder}
\end{table}

\subsection{Schematic}
    \begin{figure}[H]
        \centering
        \includegraphics[width=0.75\linewidth]{img//e_lab_2/schematic-2_3.png}
        \caption{Schematic of the ADC driver with anti-aliasing filter}
        \label{fig:Schematic-2_3}
    \end{figure}

%In figure \ref{fig:17} you can see whatever.

\subsection{Formulas and Calculations}
    To calculate the component values the general transfer function for 
    
    \begin{align}
        H(s)=\frac{H_0}{1+a \cdot s+b \cdot s^2} \label{eq:tf_lp_gen}
    \end{align}
    
    \begin{align}
        H(P)=-\frac{\dfrac{R_2}{R_1}}{1+\omega_c C_1\left(R_2+R_3+\dfrac{R_2 R_3}{R_1}\right) P+\omega_c^2 C_1 C_2 R_2 R_3 P^2} \label{eq:tf_mf_lp}
    \end{align}
    
    \begin{align}
        & H_0 = -\frac{R_2}{R_1} \\
        & a = \omega_c C_1\left(R_2+R_3+\dfrac{R_2 R_3}{R_1}\right) \\
        & b = \omega_c^2 C_1 C_2 R_2 R_3 \\
        & P = \frac{s}{\omega_c}
    \end{align}
    
    The decision was made to use a Butterworth characteristic for the second order multi-feedback filter. The values for a and b for a Butterworth filter are as follows:
    
    \begin{align}
        & a = \sqrt{2} \\
        & b = 1
    \end{align}
    
    $\omega_c$ is calculated according to equation \ref{eq:w_c}.
    
    \begin{align}
        \omega_c = 2\pi f_{cu} = 2\pi \cdot \SI{40}{kHz} = 251 \cdot 10^3 s^{-1} \label{eq:w_c}
    \end{align}
    
    \subsubsection{SNR and ENOB Calculation of the ADC Driver System}
    
    The system consists of a single-ended to differential driver implemented with the LM4562 operational amplifier, followed by the AD7626 16-bit ADC. The total system performance is limited by both the ADC quantization noise and the analog front-end noise.
    
    \subsubsection{Ideal ADC SNR}
    
    For an ideal N-bit ADC with a full-scale sinusoidal input, the theoretical
    Signal-to-Noise Ratio (SNR) is given by:
    
    \begin{equation}
    SNR_{ideal} = 6.02N + 1.76 \,\, [dB]
    \end{equation}
    
    For a 16-bit ADC:
    
    \begin{equation}
    SNR_{ideal} = 6.02 \cdot 16 + 1.76 = 98.08 \, dB
    \end{equation}
    
    This value represents the quantization noise limit only.
    
    \subsubsection{Real ADC SNR (Datasheet Value)}
    
    According to the AD7626 datasheet, the typical SNR for a full-scale sine wave is:
    
    \begin{equation}
    SNR_{ADC} \approx 95 \, dB
    \end{equation}
    
    This value includes internal non-idealities and thermal noise.
    
    \subsubsection{Analog Driver Signal Level}
    
    The input signal to the driver is:
    
    \begin{equation}
    V_{in,pp} = 100 \, mV
    \end{equation}
    
    The differential gain of the driver is:
    
    \begin{equation}
    A_{diff} = 20
    \end{equation}
    
    Therefore, the peak-to-peak voltage at the ADC input is:
    
    \begin{equation}
    V_{out,pp} = A_{diff} \cdot V_{in,pp} = 20 \cdot 0.1 = 2 \, V
    \end{equation}
    
    \subsubsection{RMS Signal Value at the ADC Input}
    
    For a sinusoidal signal:
    
    \begin{equation}
    V_{signal,RMS} = \frac{V_{pp}}{2\sqrt{2}} = \frac{2}{2\sqrt{2}} = \frac{2}{2.828} = 0.707 \, V
    \end{equation}
    
    \subsubsection{RMS Noise Voltage of the Driver}
    
    From PSpice noise simulation over the bandwidth
    (100 Hz -- 40 kHz):
    
    \begin{equation}
    V_{noise,RMS} = 40.7 \, \mu V
    \end{equation}
    
    \subsubsection{Driver SNR}
    
    \begin{equation}
    SNR_{driver} = \frac{V_{signal,RMS}}{V_{noise,RMS}} = \frac{0.707}{40.7 \times 10^{-6}} = 17375
    \end{equation}
    
    
    In decibels:
    
    \begin{equation}
    SNR_{driver,dB} = 20 \log_{10}(17375) = 84.8 \, dB
    \end{equation}
    
    
    \subsubsection{Total System SNR}
    
    Since ADC noise and driver noise are uncorrelated,
    they must be combined in linear scale:
    
    \begin{equation}
    \frac{1}{SNR_{total}} =
    \frac{1}{SNR_{ADC}} +
    \frac{1}{SNR_{driver}}
    \end{equation}
    
    First, convert both SNR values to linear scale:
    
    \begin{equation}
    SNR_{ADC,lin} = 10^{95/10} = 3.16 \times 10^9
    \end{equation}
    
    \begin{equation}
    SNR_{driver,lin} = 10^{84.8/10} = 3.02 \times 10^8
    \end{equation}
    
    
    Combine both contributions:
    
    \begin{equation}
    \frac{1}{SNR_{total,lin}} =
    \frac{1}{3.16 \times 10^9} +
    \frac{1}{3.02 \times 10^8}
    \end{equation}
    
    \begin{equation}
    SNR_{total,lin} \approx 2.76 \times 10^8
    \end{equation}
    
    Convert back to decibels:
    
    \begin{equation}
    SNR_{total,dB} =
    10 \log_{10}(2.76 \times 10^8) = 84.4 \, dB
    \end{equation}
    
    \subsubsection{Effective Number of Bits (System)}
    
    \begin{equation}
    ENOB =
    \frac{SNR_{total,dB} - 1.76}{6.02} = \frac{84.4 - 1.76}{6.02} = \frac{82.64}{6.02} = 13.7 \, bits
    \end{equation}


\subsection{Table(s) with Measurement Results}
    \begin{table}[H]
        \centering
        \caption{Component values for ADC driver and anti-aliasing filter}
        \begin{tabular}{c!{\vrule width 2pt}c!{\vrule width 2pt}c!{\vrule width 2pt}c|}
            \cline{2-4}
             & \textbf{Calculated} & \textbf{Chosen} & \textbf{Measured} 
            \\ \noalign{\hrule height 2pt}
            \multicolumn{1}{|c!{\vrule width 2pt}}{$R_1$ in \textohm} & - & 820 & 817 \\ \hline
            \multicolumn{1}{|c!{\vrule width 2pt}}{$R_2$ in k\textohm} & - & 8.2 & 8.1602 \\ \hline
            \multicolumn{1}{|c!{\vrule width 2pt}}{$R_3$ in k\textohm} & - & 2.7 & 2.6888 \\ \hline
            \multicolumn{1}{|c!{\vrule width 2pt}}{$R_4$ in k\textohm} & 10 & 10 & 9.8642 \\ \hline
            \multicolumn{1}{|c!{\vrule width 2pt}}{$R_5$ in k\textohm} & 10 & 10 & 9.8666 \\ \hline
            \multicolumn{1}{|c!{\vrule width 2pt}}{$C_1$ in pF} & - & 150 & 155 \\ \hline
            \multicolumn{1}{|c!{\vrule width 2pt}}{$C_2$ in nF} & - & 4.7 & 4.61 \\ \hline
            \multicolumn{1}{|c!{\vrule width 2pt}}{$C_3$ in \textmu F} & - & 1.5 & 1.56 \\ \hline
            \multicolumn{1}{|c!{\vrule width 2pt}}{$C_4$ in nF} & - & 100 & - \\ \hline
            \multicolumn{1}{|c!{\vrule width 2pt}}{$C_5$ in \textmu F} & - & 10 & - \\ \hline
        \end{tabular}
        \label{tab:adc_driver_components}
    \end{table}
    The measured resistor deviations are within the expected 1\% tolerance range. 
    The small deviations of the RC components slightly influence the cutoff frequency of the anti-aliasing filter, but remain within acceptable limits. 
    The measured capacitance variation of C1 (150 pF → 155 pF) leads to a minor shift of the pole frequency.
    \begin{longtable}{|c|c|c|}
        \caption{\centering Measured frequency response}
        \label{tab:frequency_response} \\
        \hline
        \textbf{Frequency / Hz} & \textbf{Gain / dB} & \textbf{Phase / Degree} \\
        \hline
        \endfirsthead
    
        \caption[]{\centering Measured frequency response (continuation)} \\
        \hline
        \textbf{Frequency / Hz} & \textbf{Gain / dB} & \textbf{Phase / Degree} \\
        \hline
        \endhead
    
        \hline
        \endfoot
    
        \hline
        \endlastfoot
    
        3      & -6.80 & 80  \\
        7      & 0.40  & 85  \\
        10     & 3.40  & 87  \\
        20     & 9.30  & 80  \\
        40     & 15.00 & 73  \\
        50     & 16.80 & 69  \\
        60     & 18.10 & 66  \\
        70     & 19.20 & 61  \\
        80     & 20.10 & 59  \\
        90     & 20.90 & 56  \\
        100    & 21.50 & 54  \\
        110    & 22.00 & 51  \\
        120    & 22.40 & 48  \\
        130    & 22.80 & 46  \\
        135    & 23.00 & 45  \\
        140    & 23.10 & 44  \\
        150    & 23.40 & 42  \\
        160    & 23.70 & 40  \\
        170    & 23.90 & 38  \\
        180    & 24.10 & 36  \\
        190    & 24.20 & 35  \\
        200    & 24.40 & 34  \\
        210    & 24.50 & 33  \\
        220    & 24.60 & 31  \\
        230    & 24.70 & 30  \\
        240    & 24.80 & 29  \\
        260    & 25.00 & 27  \\
        280    & 25.10 & 25  \\
        300    & 25.20 & 24  \\
        330    & 25.40 & 22  \\
        350    & 25.40 & 20  \\
        400    & 25.60 & 18  \\
        450    & 25.70 & 16  \\
        500    & 25.80 & 14  \\
        600    & 25.80 & 10  \\
        750    & 25.90 & 8   \\
        825    & 25.90 & 7   \\
        1000   & 26.00 & 6   \\
        1250   & 26.00 & 4   \\
        1500   & 26.00 & 2   \\
        1750   & 26.00 & 0   \\
        2250   & 26.00 & -1  \\
        2500   & 26.00 & -2  \\
        2750   & 26.00 & -3  \\
        3000   & 26.00 & -4  \\
        3500   & 26.00 & -5  \\
        6000   & 26.00 & -11 \\
        10000  & 25.90 & -18 \\
        11000  & 25.90 & -21 \\
        12000  & 25.90 & -23 \\
        13000  & 25.80 & -28 \\
        14000  & 25.80 & -30 \\
        15000  & 25.80 & -31 \\
        16000  & 25.70 & -34 \\
        17000  & 25.70 & -35 \\
        18000  & 25.60 & -37 \\
        19000  & 25.60 & -40 \\
        20000  & 25.50 & -42 \\
        22000  & 25.40 & -48 \\
        24000  & 25.20 & -54 \\
        26000  & 25.00 & -56 \\
        28000  & 24.70 & -60 \\
        30000  & 24.40 & -66 \\
        32000  & 24.10 & -72 \\
        33000  & 24.00 & -75 \\
        34000  & 23.80 & -75 \\
        35000  & 23.60 & -78 \\
        37000  & 23.20 & -80 \\
        38000  & 23.00 & -83 \\
        39000  & 22.80 & -84 \\
        40000  & 22.60 & -86 \\
        45000  & 21.50 & -93 \\
        50000  & 20.40 & -106 \\
        55000  & 19.20 & -109 \\
        60000  & 18.00 & -114 \\
        65000  & 16.90 & -120 \\
        70000  & 15.50 & -122 \\
        75000  & 14.50 & -125 \\
        100000 & 9.90  & -130 \\
        120000 & 6.80  & -140 \\
        150000 & 3.30  & -156 \\
        190000 & 0.00  & -180 \\
        200000 & -1.00 & -180 \\
    
    \end{longtable}

    The measured response shows a constant gain of approximately 26 dB within the passband. 
    The phase shift approaches \SI{-180}{\deg} at high frequencies, which confirms the expected second-order low-pass behavior.
    The cutoff frequency can be determined from the \SI{-3}{dB} drop relative to the mid-band gain.

\subsection{Curves \& Diagrams}


\subsection{Discussion of Measurement Results}
    Although the AD7626 is a 16-bit ADC with an ideal SNR of 98.08 dB,
    the system performance is limited by the analog driver noise.
    For the selected input amplitude (2 Vpp at the ADC input),
    the overall system achieves:
    
    \begin{itemize}
    \item $SNR_{total} = 84.4 \, dB$
    \item $ENOB = 13.7 \, bits$
    \end{itemize}
    
    Therefore, the effective resolution is reduced due to analog front-end noise
    and the fact that the ADC full-scale range is not fully utilized.