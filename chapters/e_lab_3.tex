\newpage
\section{Switched Capacitor Filter}

\subsection{Task Description}
    A 5th-order clock-tunable switched capacitor filter based on the LT1063 (dual $\pm$5V supply) was implemented. 
    A non-inverting preamplifier stage using the LM4562 with a gain of 100 was placed in front of the filter.
    
    The frequency response was measured for two clock frequencies (1 MHz and 500 kHz) and compared in a Bode plot. 
    The filter behavior near the clock frequency was investigated in the time domain, including possible clock feedthrough effects.
    
    The total output noise was measured using additional amplification stages, and the input-referred noise voltage was determined for different source resistances. 
    Finally, the SNR and ENOB were calculated for a full-scale sine wave.

\subsection{Schematic}
    \begin{figure}[H]
        \centering
        \includegraphics[width=0.75\linewidth]{img//e_lab_3/schematic-2_4.png}
        \caption{Schematic of the switched-capacitor filter with pre-amplifier stage.}
        \label{fig:Schematic-2_4}
    \end{figure}
    \begin{figure}[H]
        \centering
        \includegraphics[width=0.75\linewidth]{img//e_lab_4/schematic-2_5.png}
        \caption{Schematic of the low pass filter and amplifier for noise measurement.}
        \label{fig:Schematic-2_5}
    \end{figure}

\subsection{Formulas and Calculations}

\subsubsection*{Thermal Noise of the Source Resistance}

The thermal noise density generated by a resistor is given by:

\begin{equation}
e_{n,density} = \sqrt{4 k T R_s}
\end{equation}

where:
\begin{itemize}
\item $k = 1.38 \times 10^{-23}\,J/K$ (Boltzmann constant)
\item $T = 300\,K$ (room temperature)
\item $R_s$ is the source resistance
\end{itemize}

The total RMS noise voltage over a given bandwidth $B$ is:

\begin{equation}
V_{noise} = \sqrt{4 k T B R_s}
\end{equation}

\subsubsection*{Cut-off Frequency of the MAX295}

The switched-capacitor filter used in this experiment is the MAX295, which has a clock-to-cutoff frequency ratio of 50:1.

\begin{equation}
f_c = \frac{f_{CLK}}{50}
\end{equation}

For a clock frequency of 1 MHz:

\begin{equation}
f_c = \frac{1,000,000}{50} = 20,000\,Hz = 20\,kHz
\end{equation}

For a clock frequency of 500 kHz:

\begin{equation}
f_c = \frac{500,000}{50} = 10,000\,Hz = 10\,kHz
\end{equation}

\subsubsection*{Noise Bandwidth Correction}

Since the filter is an 8th-order Butterworth filter, the noise bandwidth is not equal to the cut-off frequency. A correction factor must be applied.
This correction factor can be found in \cite{winder}
For an 8th-order Butterworth filter:

\begin{equation}
B_{noise} = 1.006455 \cdot f_c
\end{equation}

For $f_c = 20\,kHz$:

\begin{equation}
B_{noise} = 1.006455 \cdot 20000 = 20129.1\,Hz
\end{equation}

\begin{equation}
\sqrt{B_{noise}} = 141.877
\end{equation}

\subsubsection*{Total System Gain}

The complete system gain is:

\begin{equation}
A_{total} = 100 \cdot 1 \cdot 10 \cdot 100 = 100000
\end{equation}

\subsubsection*{Input-Referred Noise}

The input-referred noise is calculated as:

\begin{equation}
U_{noise,in} = \frac{U_{noise,out}}{A_{total}}
\end{equation}

\subsubsection*{Noise Spectral Density}

\begin{equation}
e_n = \frac{U_{noise,in}}{\sqrt{B_{noise}}}
\end{equation}

\subsubsection*{Measured Results}

Input signal (sine wave):

\begin{equation}
V_{signal,RMS} = \frac{50\,mV_{pp}}{2\cdot\sqrt{2}} = 17.67\,mV_{RMS}
\end{equation}


\subsubsection*{Case 1: Input connected to GND}

\begin{equation}
U_{noise,out} = 99.69\,mV
\end{equation}

\begin{equation}
U_{noise,in} = \frac{99.69\,mV}{100000}
= 0.9969\,\mu V
\end{equation}

\begin{equation}
e_n = \frac{0.9969 \times 10^{-6}}{141.877}
= 7.03\,nV/\sqrt{Hz}
\end{equation}

Since we need an input signal to calculate the SNR, we assume that the input signal is 50 mVpp.
\begin{equation}
SNR = 20 \log_{10}\left(\frac{17.67\,mV}{0.9969\,\mu V}\right)= 84.97\,dB
\end{equation}

\begin{equation}
ENOB = \frac{SNR - 1.76}{6.02}
= 13.82\,bits
\end{equation}

\subsubsection*{Case 2: $R_s = 51\,\Omega$}

\begin{equation}
U_{noise,out} = 101.24\,mV
\end{equation}

\begin{equation}
U_{noise,in} = \frac{101.24\,mV}{100000}
= 1.0124\,\mu V
\end{equation}

\begin{equation}
e_n = \frac{1.0124 \times 10^{-6}}{141.877}
= 7.13\,nV/\sqrt{Hz}
\end{equation}

SNR and ENOB were calculated assuming an input sine wave of 50 mVpp (17.67 mV RMS)

\begin{equation}
SNR = 20 \log_{10}
\left(
\frac{17.67\,mV}{1.0124\,\mu V}
\right)
= 84.83\,dB
\end{equation}

\begin{equation}
ENOB = \frac{84.83 - 1.76}{6.02}
= 13.80\,bits
\end{equation}

\subsubsection*{Case 3: $R_s = 100\,k\Omega$}

\begin{equation}
U_{noise,out} = 122.2\,mV
\end{equation}

\begin{equation}
U_{noise,in} = \frac{122.2\,mV}{100000}
= 1.222\,\mu V
\end{equation}

\begin{equation}
e_n = \frac{1.222 \times 10^{-6}}{141.877}
= 8.61\,nV/\sqrt{Hz}
\end{equation}

SNR and ENOB were calculated assuming an input sine wave of 50 mVpp (17.67 mV RMS)

\begin{equation}
SNR = 20 \log_{10}
\left(
\frac{17.67\,mV}{1.222\,\mu V}
\right)
= 83.18\,dB
\end{equation}

\begin{equation}
ENOB = \frac{83.18 - 1.76}{6.02}
= 13.54\,bits
\end{equation}

\subsection{Table(s) with Measurement Results}
    \begin{table}[H]
        \centering
        \caption{Component values for switched capacitor filter}
        \begin{tabular}{c!{\vrule width 2pt}c!{\vrule width 2pt}c!{\vrule width 2pt}c|}
            \cline{2-4}
             & \textbf{Calculated} & \textbf{Chosen} & \textbf{Measured} 
            \\ \noalign{\hrule height 2pt}
            \multicolumn{1}{|c!{\vrule width 2pt}}{$R_g$ in \textohm} & 50 & 50 & 51.085 \\ \hline
            \multicolumn{1}{|c!{\vrule width 2pt}}{$R_{g2}$ in k\textohm} & 100 & 100 & 99.959 \\ \hline
            \multicolumn{1}{|c!{\vrule width 2pt}}{$R_1$ in \textohm} & - & 1000 & 994.81 \\ \hline
            \multicolumn{1}{|c!{\vrule width 2pt}}{$R_2$ in k\textohm} & - & 100 & 99.761 \\ \hline
            \multicolumn{1}{|c!{\vrule width 2pt}}{$C_1$ in nF} & - & 100 & - \\ \hline
            \multicolumn{1}{|c!{\vrule width 2pt}}{$C_2$ in nF} & - & 100 & - \\ \hline
            \multicolumn{1}{|c!{\vrule width 2pt}}{$C_3$ in nF} & - & 100 & - \\ \hline
            \multicolumn{1}{|c!{\vrule width 2pt}}{$C_4$ in nF} & - & 100 & - \\ \hline
        \end{tabular}
        \label{tab:sc_filter_components}
    \end{table}
    
    \begin{table}[H]
        \centering
        \caption{Component values for low pass filter and noise measurement amplifier}
        \begin{tabular}{c!{\vrule width 2pt}c!{\vrule width 2pt}c!{\vrule width 2pt}c|}
            \cline{2-4}
             & \textbf{Calculated} & \textbf{Chosen} & \textbf{Measured} 
            \\ \noalign{\hrule height 2pt}
            \multicolumn{1}{|c!{\vrule width 2pt}}{$R_1$ in \textohm} & - & 820 & 817 \\ \hline
            \multicolumn{1}{|c!{\vrule width 2pt}}{$R_2$ in k\textohm} & - & 8.2 & 8.1602 \\ \hline
            \multicolumn{1}{|c!{\vrule width 2pt}}{$R_3$ in k\textohm} & - & 2.7 & 2.6888 \\ \hline
            \multicolumn{1}{|c!{\vrule width 2pt}}{$R_4$ in \textohm} & 1000 & 1000 & 996.6 \\ \hline
            \multicolumn{1}{|c!{\vrule width 2pt}}{$R_5$ in k\textohm} & 100 & 100 & 99.585 \\ \hline
            \multicolumn{1}{|c!{\vrule width 2pt}}{$C_1$ in pF} & - & 150 & 155 \\ \hline
            \multicolumn{1}{|c!{\vrule width 2pt}}{$C_2$ in nF} & - & 4.7 & 4.61 \\ \hline
            \multicolumn{1}{|c!{\vrule width 2pt}}{$C_3$ in \textmu F} & - & 1.5 & 1.56 \\ \hline
            \multicolumn{1}{|c!{\vrule width 2pt}}{$C_4$ in nF} & - & 100 & - \\ \hline
            \multicolumn{1}{|c!{\vrule width 2pt}}{$C_5$ in \textmu F} & - & 10 & - \\ \hline
            \multicolumn{1}{|c!{\vrule width 2pt}}{$C_6$ in nF} & - & 100 & - \\ \hline
            \multicolumn{1}{|c!{\vrule width 2pt}}{$C_7$ in \textmu F} & - & 10 & - \\ \hline
        \end{tabular}
        \label{tab:noise_amp_components}
    \end{table}
    The measured resistor values confirm the 1\% tolerance specification. 
    Since the switched capacitor filter cutoff frequency is defined by the clock frequency rather than absolute capacitor values, the passive component tolerances have only minor influence on the filter characteristics.
    
    
    % The switched capacitor filter shows a flat passband gain of approximately 40 dB. 
    % The phase shift increases progressively until reaching −90° around 5 kHz, which indicates the dominant pole region. 
    % A phase discontinuity near 10 kHz can be observed due to phase wrapping (±180° representation). 
    % The cutoff frequency is determined at the −3 dB point relative to the passband gain.
    
    The measurements for the bode plot can be found at the appendix in table \ref{tab:sc_filter_clk500kHz} and \ref{tab:sc_filter_clk1MHz}.
    
    When reducing the clock frequency from 1 MHz to 500 kHz, the cutoff frequency of the switched capacitor filter decreases proportionally. 
    This confirms the theoretical relationship:
    
    \[
    f_c \propto f_{CLK}
    \]
    
    The measured data shows approximately half the cutoff frequency compared to the 1 MHz case, which validates the internal switched capacitor principle.
    Phase wrapping at ±180° is again visible due to measurement representation.

    \begin{table}[h]
        \centering
        \caption{Noise and Performance Summary}
        \begin{tabular}{cccccccc}
            \hline
            Adjusted &
            $f$ &
            $R_g$ &
            $U_{out,noise(rms)}$ &
            SNR &
            ENOB &
            $U_{in,noise(rms)}$ &
            $e_n$ \\
            $U_{in(pp)}$ &
            (MHz) &
            ($\Omega$) &
            (mV) &
            (dB) &
            (bits) &
            ($\mu V$) &
            (nV/$\sqrt{Hz}$) \\
            \hline

            50 & 1 & GND & 99.69 & 84.97 & 13.82 & 0.997 & 7.03 \\
            50 & 1 & 51 & 101.24 & 84.83 & 13.80 & 1.012 & 7.13 \\
            50 & 1 & 100000 & 122.20 & 83.18 & 13.54 & 1.222 & 8.61 \\

            \hline
        \end{tabular}
        \label{tab:noise_summary}
    \end{table}

\subsection{Curves \& Diagrams}
    \begin{figure}[H]
        \centering
        \includegraphics[width=0.9\linewidth, trim=0 2cm 0 0, clip]{img/e_lab_3/picture_22.png}
        \caption{\centering Output (blue) of the switched capacitor filter with the input frequency near the tuning frequency.}
        \label{fig:pic_22_output_near_clk}
    \end{figure}

    \begin{figure}[H]
        \centering
        \includegraphics[width=0.9\linewidth, trim=0 2cm 0 0, clip]{img/e_lab_3/picture_23.png}
        \caption{\centering Close up of the switched-capacitor filters output signal. Each step is one switching operation of the capacitors.}
        \label{fig:pic_23_output_close_up}
    \end{figure}

    \begin{figure}[H]
        \centering
        \includesvg[inkscapelatex=false, width=0.9\linewidth]{img/e_lab_3/EL3_bode3}
        \caption{\centering Bode plot displaying measured phase and magnitude responses of the switched capacitor filter, one with a tuning frequency of \SI{1}{MHz}, one with a tuning frequency of \SI{500}{kHz}.}
        \label{fig:bode_3}
    \end{figure}

    \begin{figure}[H]
        \centering
        \includegraphics[width=0.8\linewidth, trim=0 2cm 0 0, clip]{img/e_lab_4/picture_24_Rg_GND.png}
        \caption{\centering Measurement of the total output noise voltage with the input shorted to ground and the SC-filter tuning frequency set to \SI{1}{MHz}.}
        \label{fig:pic_24_output_noise_measurement}
    \end{figure}

    \begin{figure}[H]
        \centering
        \includegraphics[width=0.8\linewidth, trim=0 2cm 0 0, clip]{img/e_lab_4/picture_24_Rg_51.png}
        \caption{\centering Measurement of the total output noise voltage with the input 51 $\Omega$ and the SC-filter tuning frequency set to \SI{1}{MHz}.}
        \label{fig:pic_25_output_noise_measurement}
    \end{figure}

    \begin{figure}[H]
        \centering
        \includegraphics[width=0.8\linewidth, trim=0 2cm 0 0, clip]{img/e_lab_4/picture_24_Rg_100k1.png}
        \caption{\centering Measurement of the total output noise voltage with the input shorted 100k $\Omega$ and the SC-filter tuning frequency set to \SI{1}{MHz}.}
        \label{fig:pic_26_output_noise_measurement}
    \end{figure}
\subsection{Discussion of Measurement Results}

In principle, for our measurements we had to use the LT1063 switched capacitor filter, but since we did not have it in our component stock, we used the MAX295, which is an 8th order switched capacitor filter.
As we can see in Figure \ref{fig:bode_3}, the MAX295 switched capacitor filter has a magnitude response that shows an active low pass. 
The gain in the passband is approximately 40 dB, which indicates that the filter amplifies signals within its passband frequency range. 
The cutoff frequency is around 20 kHz for a tuning frequency of 1 MHz, and around 10 kHz for a tuning frequency of 500 kHz. 
This confirms the theoretical relationship between the cutoff frequency and the tuning frequency in switched capacitor filters. \cite{max295}

When the filter input frequency is close to the tuning frequency, a ‘low signal’ is observed in the oscilloscope measurement section, as shown in Figure \ref{fig:pic_22_output_near_clk}. 
This is because the input frequency is much higher than the filter cutoff frequency, resulting in significant signal attenuation. 
Figure \ref{fig:pic_23_output_close_up} shows a close-up of the output signal from the switched capacitor filter, where each step represents a capacitor switching operation. 
This is characteristic of switched capacitor filters, where the signal is sampled and processed based on the switching frequency.

Subsequently, the total output noise voltage was measured by connecting the output of the switched capacitor filter to an amplifier circuit to measure the noise.
Figure \ref{fig:pic_24_output_noise_measurement} shows the measurement of the total output noise voltage with the input connected to ground.
Figure \ref{fig:pic_25_output_noise_measurement} shows the measurement of the total output noise voltage with the 51 $\Omega$ input.
Figure \ref{fig:pic_26_output_noise_measurement} shows the measurement of the total output noise voltage with the input at 100k $\Omega$.
Table \ref{tab:noise_summary} summarises the results of the noise and performance measurements for different input resistances.

By observing these values, we can determine that when the input signal is connected to ground, we are measuring the noise voltage at the output of the entire set of amplifier stages. 
Then, as the input resistance increases, the total output noise voltage also increases. 
This is because the thermal noise generated by the input resistance increases with its value, 
resulting in higher noise voltage at the output of the switched capacitor filter.

This leads us to the conclusion that at low resistances, op-amp noise dominates, while at high resistances, thermal noise from the input resistance dominates.
