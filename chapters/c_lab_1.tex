\chapter{Computer Lab}

\section{Noise Analysis with PSpice}

\subsection{Task Description}
The task of this lab session was to conduct simulations of transistor and OpAmp circuits in PSpice, 
and to analyze the noise generated by the different components of the circuits. 
The circuits that were simulated include a single stage emitter circuit, an inverting amplifier, 
and a non-inverting amplifier. 
The noise spectral voltage density of the output of each circuit was analyzed, 
and the contribution of each individual resistor to the overall noise was also examined. 

\subsection{Schematic}
    \begin{figure}[H]
        \centering
        \includegraphics[width=0.9\linewidth]{img//c_lab_1/pic_5_emitter_schematic.png}
        \caption{Schematic of the single stage emitter circuit used for simulation.}
        \label{fig:pic_5_emitter_schematic}
    \end{figure}
    
    \begin{figure}[H]
        \centering
        \includegraphics[width=0.9\linewidth]{img//c_lab_1/pic_7_inverting_schematic.png}
        \caption{Schematic of the inverting amplifier circuit used for simulation.}
        \label{fig:pic_7_inverting_schematic}
    \end{figure}
    
    \begin{figure}[H]
        \centering
        \includegraphics[width=0.9\linewidth]{img//c_lab_1/pic_7_non_inverting_schematic.png}
        \caption{Schematic of the non-inverting amplifier circuit used for simulation.}
        \label{fig:pic_7_non_inverting_schematic}
    \end{figure}


\subsection{Curves \& Diagrams}

    \begin{figure}[H]
        \centering
        \includesvg[inkscapelatex=false, width=0.7\linewidth]{img/c_lab_1/pic_6_emitter_noise}
        \caption{\centering Noise spectral voltage density of the single stage emitter circuits output. The simulation results of the noise generated by each individual resistor is also depicted.}
        \label{fig:pic_6_noise_emitter}
    \end{figure}
    
    \begin{figure}[H]
        \centering
        \includesvg[inkscapelatex=false,width=0.7\linewidth]{img/c_lab_1/pic_8_inverting_noise}
        \caption{\centering Noise spectral voltage density of the inverting amplifiers output. The simulation results of the noise generated by each individual resistor is also depicted.}
        \label{fig:pic_8_noise_inverting}
    \end{figure}
    
    \begin{figure}[H]
        \centering
        \includesvg[inkscapelatex=false,width=0.7\linewidth]{img/c_lab_1/pic_8_non_inverting_noise}
        \caption{\centering Noise spectral voltage density of the non-inverting amplifiers output. The simulation results of the noise generated by each individual resistor is also depicted.}
        \label{fig:pic_8_noise_non_inverting}
    \end{figure}

\subsection{Discussion of Measurement Results}
The measured noise spectral voltage density of the emitter follower stage (figure \ref{fig:pic_6_noise_emitter}) shows a 
similar trend as the emitter followers magnitude response. The emitter follower effectively acts as a bandpass-filter with a very wide pass band. 
The lower corner frequency of the pass band is determined by the value of the input coupling capacitor and the upper corner frequency by the transistors parasitic capacitances. 
The thermal noise originating from the resistors is amplified according to the passband characteristics of the emitter follower. R4 and R5 are exceptions because they are not affected 
by the input coupling capacitor and therefore have no lower corner frequency. R1 is the biggest contributor to the thermal noise eventhough it has a relatively low resistance because 
it's noise acts as a input signal to the amplifier and is therefore amplified by the gain of the emitter follower.

The shot noise, caused by the transistor, is also visible at lower frequencies. It can be seen when looking at the noise of R3, R4 and the total output noise. 

The inverting and non-inverting amplifier stages show an almost identical noise spectral voltage density (figures \ref{fig:pic_8_noise_inverting} and \ref{fig:pic_8_noise_non_inverting}). 
The same behavior observed in the emitter follower where the noise spectal density behaves similar to the magnitude 
response of the amplifier can also be seen in these two OpAmp amplifier stages. The low-pass filter behavior of the amplifiers can be observed in the noise spectrum. 
The OpAmps shot noise is also visiable at lower frequencies and can only evident in the total noise and is not present in the individual resistors noise curves because the shot noise is generated by the OpAmp itself and not by the resistors.